Very roughly speaking, a principal fibre bundle is a bundle whose typical fibre is a Lie group. Principal fibre bundles are so immensely important because they allow us to understand any fibre bundle with fibre $F$ on which a Lie group $G$ acts. These are then called associated fibre bundles, and will be discussed later on.

This is a good time to review our earlier section on bundles. We can specialize our definition of bundle to define a \emph{smooth bundle}, which is just a bundle $(E,\pi,M)$ where $E$ and $M$ are smooth manifolds and the projection $\pi\cl E\to M$ is smooth. Two smooth bundles $(E,\pi,M)$ and $(E',\pi',M')$ are isomorphic if there exist diffeomorphisms $u,f$ such that the following diagram commutes
\bse
\begin{tikzcd}
E \ar[dd,"\pi"'] \ar[rr,"u"] && E' \ar[dd,"\pi'"]\\
&&\\
M\ar[rr,"f"] && M'
\end{tikzcd}
\ese

\section{Principal Fibre Bundles}

\bd
Let $G$ be a Lie group. A smooth bundle $(E,\pi,M)$ is called a \emph{principal $G$-bundle}\index{principal bundle} if $E$ is equipped with a free right $G$-action and
\bse
\begin{tikzcd}
E \ar[d,"\pi"'] \\
M 
\end{tikzcd}
\ \ \cong_{\mathrm{bdl}}
\begin{tikzcd}
  E \ar[d,"\rho"]\\
  E/G
\end{tikzcd}
\ese
where $\rho$ is the quotient map, defined by sending each $p\in E$ to its equivalence class (i.e.\ orbit) in the orbit space $E/G$.
\ed
Observe that since the right action of $G$ on $E$ is free, for each $p\in E$ we have
\bse
\preim_\rho(G_p) = G_p \cong_{\mathrm{diff}} G.
\ese
We said at beginning that, roughly speaking, a principal bundle is a bundle whose fibre at each point is a Lie group. Note that the formal definition is that a principal $G$-bundle is a bundle which is isomorphic to a bundle whose fibres are the orbits under the right action of $G$, which are themselves isomorphic to $G$ since the action is free.

\br
A slight generalisation would be to consider smooth bundles $E\xrightarrow{\,\pi\,}M$ where $E$ is equipped with a right $G$-action which is free and transitive on each fibre of $E\xrightarrow{\,\pi\,}M$. The isomorphism in our definition enforces the fibre-wise transitivity since $G$ acts transitively on each $G_p$ by the definition of orbit.
\er

\be
\ben[label=\alph*)]
\item Let $M$ be a smooth manifold. Consider the space
\bse
L_pM := \{(e_1,\ldots,e_{\dim M})\mid e_1,\ldots,e_{\dim M} \text{ is a basis of }T_pM\} \cong_{\mathrm{vec}} \GL(\dim M,\R).
\ese
We know from linear algebra that the bases of a vector space are related to each other by invertible linear transformations. Hence, we have 
\bse
L_pM \cong_{\mathrm{vec}} \GL(\dim M,\R).
\ese
We define the frame bundle of $M$ as
\bse
LM := \coprod_{p\in M} L_pM
\ese
with the obvious projection map $\pi\cl LM \to M$ sending each basis $(e_1,\ldots,e_{\dim M})$ to the unique point $p\in M$ such that $(e_1,\ldots,e_{\dim M})$ is a basis of $T_pM$.
By proceeding similarly to the case of the tangent bundle, we can equip $LM$ with a smooth structure inherited from that of $M$. We then find
\bse
\dim LM = \dim M + \dim T_pM = \dim M + (\dim M)^2.
\ese
\item We would now like to make $LM \xrightarrow{\,\pi\,}M$ into a principal $\GL(\dim M,\R)$-bundle. We define a right $\GL(\dim M,\R)$-action on $LM$ by
\bse
(e_1,\ldots,e_{\dim M}) \racts g := (g^a_{\phantom{a}1}e_a,\ldots,g^a_{\phantom{a}\dim M}e_a),
\ese
where $g^a_{\phantom{a}b}$ are the components of the endomorphism $g\in \GL(\dim M, \R)$ with respect to the standard basis on $\R^n$. Note that if $(e_1,\ldots,e_{\dim M})\in L_pM$, we must also have $(e_1,\ldots,e_{\dim M}) \racts g\in L_pM$. This action is free since
\bse
(e_1,\ldots,e_{\dim M}) \racts g = (e_1,\ldots,e_{\dim M})  \Leftrightarrow  (g^a_{\phantom{a}1}e_a,\ldots,g^a_{\phantom{a}\dim M}e_a)= (e_1,\ldots,e_{\dim M}) 
\ese
and hence, by linear independence, $g^a_{\phantom{a}b}=\delta^a_b$, so $g=\id_{\R^n}$. Note that since all bases of each $T_pM$ are related by some $g\in \GL(\dim M,\R)$, $\racts$ is also fibre-wise transitive. 
\item We now have to show that
\bse
\begin{tikzcd}
LM \ar[d,"\pi"'] \\
M 
\end{tikzcd}
\ \ \cong_{\mathrm{bdl}}
\begin{tikzcd}
  LM \ar[d,"\rho"]\\
  LM \big/ \GL(\dim M,\R)
\end{tikzcd}
\ese
i.e.\ that there exist smooth maps $u$ and $f$ such that the diagram
\bse
\begin{tikzcd}
LM \ar[rr,shift left,"u"] \ar[dd,"\pi"']&& LM \ar[ll,shift left,"u^{-1}"]\ar[dd,"\rho"]\\
&&\\
M \ar[rr,shift left,"f"]&&  LM \big/ \GL(\dim M,\R)\ar[ll,shift left,"f^{-1}"]
\end{tikzcd}
\ese
commutes. We can simply choose $u=u^{-1}=\id_{LM}$, while we define $f$ as
\bi{rrCl}
f \cl & M & \to &  LM\big/ \GL(\dim M,\R)\\[2pt]
& p & \mapsto & \GL(\dim M,\R)_{(e_1,\ldots,e_{\dim M})},
\ei
where $(e_1,\ldots,e_{\dim M})$ is some basis of $T_pM$, i.e.\ $(e_1,\ldots,e_{\dim M})\in \preim_\pi(\{p\})$. Note that $f$ is well-defined since every basis of $T_pM$ gives rise to the same orbit in the orbit space $LM\big/\GL(\dim M,\R)$. Moreover, it is injective since
\bse
f(p)=f(p')\ \Leftrightarrow \ \GL(\dim M,\R)_{(e_1,\ldots,e_{\dim M})} = \GL(\dim M,\R)_{(e'_1,\ldots,e'_{\dim M})},
\ese
which is true only if $(e_1,\ldots,e_{\dim M})$ and $(e'_1,\ldots,e'_{\dim M})$ are basis of the same tangent space, so $p=p'$. It is clearly surjective since every orbit in $LM\big/ \GL(\dim M,\R)$ is the orbit of some basis of some tangent space $T_pM$ at some point $p\in M$. The inverse map is given explicitly by
\bi{rrCl}
f^{-1} \cl & LM\big/ \GL(\dim M,\R) & \to & M \\[2pt]
& \GL(\dim M,\R)_{(e_1,\ldots,e_{\dim M})} & \mapsto & \pi((e_1,\ldots,e_{\dim M})).
\ei
Finally, we have
\bse
(\rho\circ\id_{LM})(e_1,\ldots,e_{\dim M}) = \GL(\dim M,\R)_{(e_1,\ldots,e_{\dim M})} = (f\circ \pi)(e_1,\ldots,e_{\dim M})
\ese
and thus $LM\xrightarrow{\,\pi\,}M$ is a principal $G$-bundle, called the \emph{frame bundle}\index{frame bundle} of $M$.
\een
\ee

\br
A note to the careful reader. As we have just done in the previous example, in the following we will sometimes simply assume that certain maps are smooth, instead of rigorously proving it. 
\er

\section{Principal Bundle Morphisms}

Recall that a bundle morphism (also called simply a bundle map) between two bundles $(E,\pi,M)$ and $(E',\pi',M')$ is a pair of maps $(u,f)$ such that the diagram
\bse
\begin{tikzcd}
E \ar[dd,"\pi"'] \ar[rr,"u"] && E' \ar[dd,"\pi'"]\\
&&\\
M\ar[rr,"f"] && M'
\end{tikzcd}
\ese
commutes, that is, $f\circ \pi = \pi' \circ u$.
\bd
Let $(P,\pi,M)$ and $(Q,\pi',N)$ be principal $G$-bundles. A \emph{principal bundle morphism} from $(P,\pi,M)$ to $(Q,\pi',N)$ is a pair of smooth maps $(u,f)$ such that the diagram
\bse
\begin{tikzcd}
P \ar[rr,"u"]&& Q \\
&&\\
P \ar[uu,"{}\racts G"] \ar[dd,"\pi"'] \ar[rr,"u"] && Q\ar[uu,"{}\blacktriangleleft G"'] \ar[dd,"\pi'"]\\
&&\\
M\ar[rr,"f"] && N
\end{tikzcd}
\ese
commutes, that is for all $p\in P$ and $g\in G$, we have
\bi{rCl}
(f\circ \pi)(p)  & = & (\pi'\circ u)(p)\\
u(p\racts g) & = & u(p)\blacktriangleleft g.
\ei
\ed
Note that $P\xrightarrow{\quad \racts G \quad } P$ is a shorthand for the inclusion of $P$ into the product $P\times G$ followed by the right action $\racts$, i.e.\
\bse
\begin{tikzcd}
P \ar[rr,"\racts G"] && P
\end{tikzcd}
\ \quad =\quad \ 
\begin{tikzcd}
P \ar[rr,"i_1"] && P\times G \ar[rr,"\racts"] && P
\end{tikzcd}
\ese
and similarly for $Q\xrightarrow{\quad \blacktriangleleft G \quad } Q$.

\bd
A principal bundle morphism between two principal $G$-bundles is an \emph{isomorphism} (or \emph{diffeomorphism}) \emph{of principal bundles} if it is also a bundle isomorphism.
\ed

\br
Note that the passage from principal bundle morphism to principal bundle isomorphism does not require any extra condition involving the Lie group $G$. We will soon see that this is because the two bundles are both principal $G$-bundles. We can further generalise the notion of principal bundle morphism as follows.
\er

\bd
Let $(P,\pi,M)$ be a principal $G$-bundle, let $(Q,\pi',N)$ be a principal $H$-bundle, and let $\rho\cl G \to H$ be a Lie group homomorphism. A \emph{principal bundle morphism}\index{principal bundle morphism} from $(P,\pi,M)$ to $(Q,\pi',N)$ is a pair of smooth maps $(u,f)$ such that the diagram
\bse
\begin{tikzcd}
P \ar[rr,"u"]&& Q \\
&&\\
P\times G\ar[uu,"{}\racts "]  \ar[rr,"u\times \rho"]&& Q\times H \ar[uu,"{}\blacktriangleleft"'] \\
&&\\
P \ar[uu,"i_1"] \ar[dd,"\pi"'] \ar[rr,"u"] && Q\ar[uu,"i_1"'] \ar[dd,"\pi'"]\\
&&\\
M\ar[rr,"f"] && N
\end{tikzcd}
\ese
commutes, that is $f \circ \pi=\pi'\circ u $ and $u$ is a $\rho$-equivariant map
\bi{rCl}
\forall \, p\in P : \forall \, g \in G : \ u(p\racts g) & = & u(p)\blacktriangleleft \rho(g).
\ei
\ed

\bd
A principal bundle morphism between principal $G$-bundle and a principal $H$-bundle is an \emph{isomorphism} (or \emph{diffeomorphism}) \emph{of principal bundles}\index{isomorphism!of principal bundles} if it is also a bundle isomorphism and $\rho$ is a Lie group isomorphism.
\ed

\bl
Let $(P,\pi,M)$ and $(Q,\pi',M)$ be principal $G$-bundles over the same base manifold $M$. Then, any $u\cl P \to Q$ such that $(u,\id_M)$ is a principal bundle morphism is necessarily a diffeomorphism.
\bse
\begin{tikzcd}
P \ar[rr,"u"]&& Q \\
&&\\
P \ar[uu,"\racts G"] \ar[ddr,"\pi"'] \ar[rr,"u"] && Q\ar[uu,"\blacktriangleleft G"'] \ar[ddl,"\pi'"]\\
&&\\
 &M& 
\end{tikzcd}
\ese
\el

\bq
We already know that $u$ is smooth since $(u,\id_M)$ is assumed to be a principal bundle morphism. It remains to check that $u$ is bijective and its inverse is also smooth.
\ben[label=\roman*)]
\item Let $p_1,p_2\in P$ be such that $u(p_1)=u(p_2)$. Then
\bse
\pi(p_1) = \pi'(u(p_1)) = \pi'(u(p_2)) = \pi(p_2),
\ese
that is, $p_1$ and $p_2$ belong to the same fibre. As the action of $G$ on $P$ is fibre-wise transitive, there is a unique $g\in G$ such that $p_1 = p_2\racts g$. Then
\bse
u(p_1)  = u(p_2\racts g)  = u(p_2) \blacktriangleleft g = u(p_1) \blacktriangleleft g,
\ese
so $g\in S_{u(p_1)}$, but since $\blacktriangleleft$ is free, we have $g=e$ and thus
\bse
p_1 = p_2\racts e = p_2.
\ese
Therefore $u$ is injective.
\item Let $q\in Q$. Choose some $p\in \preim_\pi(\pi'(q))$. Then, we have
\bse
\pi'(u(p)) = \pi(p) = \pi'(q)
\ese
so that $u(p)$ and $q$ belong to the same fibre. Hence, there is a unique $g\in G$ such that $q = u(p)\blacktriangleleft g$. We thus have
\bse
q = u(p)\blacktriangleleft g = u(p\racts g)
\ese
and since $p\racts g\in P$, the map $u$ is surjective. 
\een
Hence, $u$ is a diffeomorphism.
\eq

\bd
A principal $G$-bundle $(P,\pi,M)$ if it is called \emph{trivial} if it is isomorphic as a principal $G$-bundle to the principal $G$-bundle $(M\times G,\pi_1,M)$ where $\pi_1$ is the projection onto the first component and the action is defined as
\bi{rrCl}
\blacktriangleleft \cl & (M\times G) \times G & \to &M\times G\\
& ((p,g),g') & \mapsto & (p,g)\blacktriangleleft g' := (p,g\bullet g').
\ei
\ed
By the previous lemma, a principal $G$-bundle $(P,\pi,M)$ is trivial if there exists a smooth map $u\cl P\to M\times G$ such that the following diagram commutes.
\bse
\begin{tikzcd}
P \ar[rr,"u"]&& M\times G \\
&&\\
P \ar[uu,"{}\racts G"] \ar[ddr,"\pi"'] \ar[rr,"u"] && M\times G\ar[uu,"{}\blacktriangleleft G"'] \ar[ddl,"\pi_1"]\\
&&\\
& M & 
\end{tikzcd}
\ese

The following result provides a necessary and sufficient criterion for when a principal bundle is trivial. Note that while we have used the lower case letter $p$ almost exclusively to denote points of the base manifold $M$, in the next proof we will use it to denote points of the total space $P$ instead.

\begin{theorem}
A principal $G$-bundle $(P,\pi,M)$ is trivial if, and only if, there exists a smooth section $\sigma\in\Gamma(P)$, that is, a smooth $\sigma \cl M \to P$ such that $\pi\circ \sigma = \id_M$.
\end{theorem}

\bq
\begin{itemize}
\item[$(\Rightarrow)$] Suppose that $(P,\pi,M)$ is trivial. Then there exists a diffeomorphism $u\cl P \to M\times G$ which make the following diagram commute
\bse
\begin{tikzcd}
P && \\
&&\\
P \ar[uu,"{}\racts G"] \ar[ddr,"\pi"']  && M \ar[ll,"u^{-1}"'] \times G\ar[ddl,"\pi_1"]\\
&&\\
& M & 
\end{tikzcd}
\ese
We can define
\bi{rrCl}
\sigma \cl & M & \to & P\\
& m & \mapsto & u^{-1}(m,e),
\ei
where $e$ is the identity of $G$. Then $\sigma$ is smooth since it is the composition of $u^{-1}$ with the map $p\mapsto (p,e)$, which are both smooth. We also have
\bse
(\pi\circ\sigma)(pm=\pi( u^{-1}(m,e)) = \pi_1 (m,e) = m,
\ese
for all $m\in M$, hence $\pi\circ\sigma=\id_M$ and thus $\sigma\in \Gamma(P)$.

\item[$(\Leftarrow)$] Suppose that there exists a smooth section $\sigma\cl M\to P$. Let $p\in P$ and consider the point $\sigma(\pi(p))\in P$. We have
\bse
\pi(\sigma(\pi(p))) = \id_M(\pi(p)) =\pi(p),
\ese
hence $\sigma(\pi(p))$ and $p$ belong to the same fibre, and thus there exists a unique group element in $G$ which links the two points via $\racts$. Since this element depends on both $\sigma$ and $p$, let us denote it by $\chi_\sigma(p)$. Then, $\chi_\sigma$ defines a function
\bi{rrCl}
\chi_\sigma \cl & P & \to & G\\
& p & \mapsto & \chi_\sigma(p)
\ei
and we can write
\bse
\forall \, p \in P : \ p = \sigma(\pi(p))\racts \chi_\sigma(p).
\ese
In particular, for any other $g\in G$ we have $p\racts g\in P$ and thus
\bse
p\racts g = \sigma(\pi(p\racts g))\racts \chi_\sigma(p\racts g) = \sigma(\pi(p))\racts \chi_\sigma(p\racts g),
\ese
where the second equality follows from the fact that the fibres of $P$ are precisely the orbits under the action of $G$.

On the other hand, we can act on the right with an arbitrary $g\in G$ directly to obtain  

\bse
p\racts g = (\sigma(\pi(p))\racts \chi_\sigma(p))\racts g = \sigma(\pi(p))\racts (\chi_\sigma(p)\bullet g).
\ese
Combining the last two equations yields
\bse
 \sigma(\pi(p))\racts \chi_\sigma(p\racts g) = \sigma(\pi(p))\racts (\chi_\sigma(p)\bullet g)
\ese
and hence
\bse
 \chi_\sigma(p\racts g) = (\chi_\sigma(p)\bullet g).
\ese
We can now define the map
\bi{rrCl}
u_\sigma \cl & P & \to & M \times G\\
& p & \mapsto & (\pi(p),\chi_\sigma(p)).
\ei
By our previous lemma, it suffices to show that $u_\sigma$ is a principal bundle map.

\bse
\begin{tikzcd}
P \ar[rr,"u_\sigma"]&& M\times G \\
&&\\
P \ar[uu,"{}\racts G"] \ar[ddr,"\pi"'] \ar[rr,"u_\sigma"] && M\times G\ar[uu,"{}\blacktriangleleft G"'] \ar[ddl,"\pi_1"]\\
&&\\
& M & 
\end{tikzcd}
\ese
By definition, we have
\bse
(\pi_1\circ u_\sigma)(p) = \pi_1 (\pi(p),\chi_\sigma(p))=\pi(p)
\ese
for all $p\in P$, so the lower triangle commutes. Moreover, we have
\bi{rCl}
u_\sigma(p\racts g) & = & (\pi(p\racts g),\chi_\sigma(p\racts g))\\
 & = & (\pi(p),\chi_\sigma(p)\bullet g))\\
 & = & (\pi(p),\chi_\sigma(p))\blacktriangleleft g\\
 & = & u_\sigma(p)\blacktriangleleft g
\ei
for all $p\in P$ and $g\in G$, so the upper square also commutes and hence $(P,\pi,M)$ is a trivial bundle. \qedhere
\end{itemize}
\eq

\be
The existence of a section on the frame bundle $LM$ can be reduced to the existence of $(\dim M)$ non-everywhere vanishing linearly independent vector fields on $M$. Since no such vector field exists on even-dimensional spheres, $LS^{2n}$ is always non-trivial.
\ee

An associated fibre bundle is a fibre bundle which is associated (in a precise sense) to a principal $G$-bundle. Associated bundles are related to their underlying principal bundles in a way that models the transformation law for components under a change of basis.

\subsection{Associated Fibre Bundles}

\bd
Let $(P,\pi,M)$ be a principal $G$-bundle and let $F$ be a smooth manifold equipped with a left $G$-action $\lacts$. We define
\ben[label=\roman*)]
\item $P_F:=(P\times F)/{\sim_G}$, where $\sim_G$ is the equivalence relation
\bse
(p,f) \sim_G (p',f') \quad :\Leftrightarrow \quad \exists \, g\in G : \biggl\{ \ba{rcl} p' & = & p\racts g \\ f' & = & g^{-1} \lacts f \ea 
\ese
We denote the points of $P_F$ as $[p,f]$.
\item The map 
\bi{rrCl}
\pi_F\cl & P_F & \to & M\\
& [p,f] & \mapsto & \pi(p),
\ei
which is well-defined since, if $[p',f']=[p,f]$, then for some $g\in G$
\bse
\pi_F([p',f']) = \pi_F([p\racts g,g^{-1} \lacts f]):=\pi(p\racts g)=\pi(p)=:\pi_F([p,f]) .
\ese
\een
The \emph{associated bundle}\index{associated bundle} (to $(P,\pi,M)$, $F$ and $\lacts$) is the bundle $(P_F,\pi_F,M)$.
\ed

\be
Recall that the frame bundle $(LM,\pi,M)$ is a principal $\GL(d,\R)$-bundle, where $d=\dim M$, with right $G$-action $\racts\cl LM \times G \to LM$ given by
\bse
(e_1,\ldots,e_{d}) \racts g := (g^a_{\phantom{a}1}e_a,\ldots,g^a_{\phantom{a}d}e_a).
\ese
Let $F:= \R^{d}$ (as a smooth manifold) and define a left action
\bi{rrCl}
\lacts \cl & \GL(d,\R) \times \R^{d} & \to & \R^{d}\\
& (g,x) & \mapsto & g\lacts x,
\ei
where 
\bse
(g\lacts x)^a := g^a_{\phantom{a}b} x^b.
\ese
Then $(LM_{\R^d},\pi_{\R^d},\R^d)$ is the associated bundle. In fact, we have a bundle isomorphism

\bse
\begin{tikzcd}
LM_{\R^d} \ar[dd,"\pi_{\R^d}"'] \ar[rr,"u"] && TM\ar[dd,"\pi"] \\
&& \\
M \ar[rr,"\id_M"] && M
\end{tikzcd}
\ese
where $(TM,\pi,M)$ is the tangent bundle of $M$, and $u$ is defined as
\bi{rcCl}
u \cl & LM_{\R^d} & \to & TM\\
& [(e_1,\ldots,e_d),x] & \mapsto & x^ae_a.
\ei
The inverse map $u^{-1}\cl TM \to LM_{\R^d}$ works as follows. Given any $X\in TM$, pick any basis $(e_1,\ldots,e_d)$ of the tangent space at the point $\pi(X)\in M$, i.e.\ any element of $L_{\pi(X)}M$. Decompose $X$ as $x^ae_a$, with each $x^a\in \R$, and define
\bse
u^{-1}(X) := [(e_1,\ldots,e_d),x].
\ese
The map $u^{-1}$ is well-defined since, while the pair $((e_1,\ldots,e_d),x)\in LM\times \R^d$ clearly depends on the choice of basis, the equivalence class 
\bse
[(e_1,\ldots,e_d),x]\in LM_{\R^d}:=(LM\times \R^d)/{\sim_G}
\ese
does not. It includes all pairs $((e_1,\ldots,e_d)\racts g,g^{-1}\lacts x)$ for every $g\in \GL(d,\R)$, i.e.\ every choice of basis together with the ``right'' components $x\in \R^d$.
\ee

\br
Even though the associated bundle $(LM_{\R^d},\pi_{\R^d},\R^d)$ is isomorphic to the tangent bundle $(TM,\pi,M)$, note a subtle difference between the two. On the tangent bundle, the transformation law for a change of basis and the related transformation law for components are \emph{deduced} from the definitions by undergraduate linear algebra.

On the other hand, the transformation laws on $LM_{\R^d}$ were \emph{chosen} by us in its definition. We chose the Lie group $\GL(d,\R)$, the specific right action $\racts$ on $LM$, the space $\R^d$, and the specific left action on $\R^d$. It just happens that, with these choices, the resulting associated bundle is isomorphic to the tangent bundle. Of course, we have the freedom to make different choices and construct bundles which behave very differently from $TM$. 
\er

\be
Consider the principal $\GL(d,\R)$-bundle $(LM,\pi,M)$ again, with the same right action as before. This time we define
\bse
F:= (\R^d)^{\times p}\times({\R^d}^*)^{\times q} := \underbrace{\R^d\times\cdots\times\R^d}_{p \text{ times}}\times \underbrace{{\R^d}^*\times\cdots\times{\R^d}^*}_{q \text{ times}} 
\ese
with left $\GL(d,\R)$-action $\lacts\cl\GL(d,\R)\times F \to F$ given by
\bse
(g\lacts f)^{a_1\cdots a_p}_{\phantom{a_1\cdots a_p}b_1\cdots b_q} := g^{a_1}_{\phantom{a_1}\widetilde a_1} \cdots g^{a_p}_{\phantom{a_p}\widetilde a_p} (g^{-1})^{\widetilde b_1}_{\phantom{b_1}b_1}\cdots  (g^{-1})^{\widetilde b_q}_{\phantom{b_q}b_q} \, f^{\widetilde a_1\cdots \widetilde a_p}_{\phantom{a_1\cdots a_p}\widetilde b_1\cdots \widetilde b_q}.
\ese
Then, the associated bundle $(LM_F,\pi_F,M)$ thus constructed is isomorphic to $(T^p_qM,\pi,M)$, the $(p,q)$-tensor bundle on $M$.
\ee

Now for something new, consider the following.

\bd
Let $M$ be a smooth manifold and let $(LM,\pi,M)$ be its frame bundle, with right $\GL(d,\R)$-action as above. Let $F:= (\R^d)^{\times p}\times({\R^d}^*)^{\times q}$ and define a left $\GL(d,\R)$-action on $F$ by
\bse
(g\lacts f)^{a_1\cdots a_p}_{\phantom{a_1\cdots a_p}b_1\cdots b_q} := (\det g^{-1})^\omega\,g^{a_1}_{\phantom{a_1}\widetilde a_1} \cdots g^{a_p}_{\phantom{a_p}\widetilde a_p} (g^{-1})^{\widetilde b_1}_{\phantom{b_1}b_1}\cdots  (g^{-1})^{\widetilde b_q}_{\phantom{b_q}b_q} \, f^{\widetilde a_1\cdots \widetilde a_p}_{\phantom{a_1\cdots a_p}\widetilde b_1\cdots \widetilde b_q},
\ese
where $\omega\in \Z$. Then the associated bundle $(LM_F,\pi_F,M)$ is called the \emph{$(p,q)$-tensor $\omega$-density bundle} on $M$, and its sections are called \emph{$(p,q)$-tensor densities of weight}\index{tensor density} $\omega$.
\ed

\br
Some special cases include the following.
\ben[label=\roman*)]
\item If $\omega = 0$, we recover the $(p,q)$-tensor bundle on $M$.
\item If $F=\R$ (i.e.\ $p=q=0$), the left action reduces to
\bse
(g\lacts f) = (\det g^{-1})^\omega\, f,
\ese
which is the transformation law for a \emph{scalar density of weight $\omega$}.
\item If $\GL(d,\R)$ is restricted in such a way that we always have $(\det g^{-1})=1$, then tensor densities are indistinguishable from ordinary tensor fields. This is why you probably haven't met tensor densities in your special relativity course.
\een
\er

\be
Recall that if $B$ is a bilinear form on a $K$-vector space $V$, the determinant of $B$ is not independent from the choice of basis. Indeed, if $\{e_a\}$ and $\{e'_b:=g^a_{\phantom{a}b}e_a\}$ are both basis of $V$, where $g\in \GL(\dim V,K)$, then
\bse
(\det B)' = (\det g^{-1})^2\det B.
\ese
Once recast in the principal and associated bundle formalism, we find that the determinant of a bilinear form is a scalar density of weight $2$.
\ee

\section{Associated Bundle Morphisms}

\bd
Let $(P_F,\pi_F,M)$ to $(Q_F,\pi'_F,N)$ be the associated bundles (with the same fibre $F$) of two principal $G$-bundles $(P,\pi,M)$ and $(Q,\pi',N)$. An \emph{associated bundle map} between the associated bundles is a bundle map $(\widetilde u,v)$ between them such that for some $u$, the pair $(u,v)$ is a principal bundle map between the underlying principal $G$-bundles and  
\bse
\widetilde u([p,f]) := [u(p),f].
\ese
Equivalently, the following diagrams both commute.
\bse
\begin{tikzcd}
P_F \ar[dd,"\pi_F"'] \ar[rr,"\widetilde u"] && Q_F\ar[dd,"\pi_F'"]\\
&&\\
M \ar[rr,"v"] && N 
\end{tikzcd}
\qquad \quad
\begin{tikzcd}
P \ar[rr,"u"]&& Q \\
&&\\
P \ar[uu,"\racts G"] \ar[dd,"\pi"'] \ar[rr,"u"] && Q\ar[uu,"\blacktriangleleft G"'] \ar[dd,"\pi'"]\\
&&\\
M \ar[rr,"v"]&& N
\end{tikzcd}
\ese
\ed

\bd
An associated bundle map $(\widetilde u,v)$ is an \emph{associated bundle isomorphism}\index{isomorphism!of associated bundles} if $\widetilde u$ and $v$ are invertible and $(\widetilde u^{-1},v^{-1})$ is also an associated bundle map.
\ed

\br
Note that two associated $F$-fibre bundles may be isomorphic as bundles but not as associated bundles. In other words, there may exist a bundle isomorphism between them, but there may not exist any bundle isomorphism between them which can be written as in the definition for some principal bundle isomorphism between the underlying principal bundles. 
\er

\noindent Recall that an $F$-fibre bundle $(E,\pi,M)$ is called trivial if there exists a bundle isomorphism
\bse
\begin{tikzcd}
F \ar[rr] && E\ar[ddr,"\pi"]\ar[rr,"u"] && M\times F \ar[ddl,"\pi_1"]\\
&&&&\\
&&& M &
\end{tikzcd}
\ese
while a principal $G$-bundle is called trivial if there exists a principal bundle isomorphism
\bse
\begin{tikzcd}
P \ar[rr,"u"]&& M\times G \\
&&\\
P \ar[uu,"\racts G"] \ar[ddr,"\pi"'] \ar[rr,"u"] && M\times G\ar[uu,"\blacktriangleleft G"'] \ar[ddl,"\pi_1"]\\
&&\\
& M & 
\end{tikzcd}
\ese

\bd
An associated bundle $(P_F,\pi_F,M)$ is called \emph{trivial} if the underlying principal $G$-bundle $(P,\pi,M)$ is trivial.
\ed

\bp
A trivial associated bundle is a trivial fibre bundle.
\ep
Note that the converse does not hold. An associated bundle can be trivial as a fibre bundle but not as an associated bundle, i.e.\ the underlying principal fibre bundle need not be trivial simply because the associated bundle is trivial as a fibre bundle.

\bd
Let $H$ be a closed Lie subgroup of $G$. Let $(P,\pi,M)$ be a principal $H$-bundle and $(Q,\pi',M)$ a principal $G$-bundle. If there exists a principal bundle map from $(P,\pi,M)$ to $(Q,\pi',M)$, i.e.\ a smooth bundle map which is equivariant with respect to the inclusion of $H$ into $G$, then $(P,\pi,M)$ is called an \emph{$H$-restriction} of $(Q,\pi',M)$, while $(Q,\pi',M)$ is called a \emph{$G$-extension} of $(P,\pi,M)$.  
\ed

\bt
Let $H$ be a closed Lie subgroup of $G$.
\ben[label=\roman*)]
\item Any principal $H$-bundle can be extended to a principal $G$-bundle.
\item A principal $G$-bundle $(P,\pi,M)$ can be restricted to a principal $H$-bundle if, and only if, the bundle $(P/H,\pi',M)$ has a section.
\een
\et

\be
\ben[label=\roman*)]
\item The bundle $(LM/\SO(d),\pi,M)$ always has a section, and since $\SO(d)$ is a closed Lie subgroup of $\GL(d,\R)$, the frame bundle can be restricted to a principal $\SO(d)$-bundle. This is related to the fact that any manifold can be equipped with a Riemannian metric.   
\item The bundle $(LM/\SO(1,d-1),\pi,M)$ may or may not have a section.  For example, the bundle $(LS^2/\SO(1,1),\pi,S^2)$ does not admit any section, and hence we cannot restrict $(LS^2/\SO(1,1),\pi,S^2)$ to a principal $\SO(1,1)$-bundle, even though $\SO(1,1)$ is a closed Lie subgroup of $\GL(2,\R)$. This is related to the fact that the $2$-sphere cannot be equipped with a Lorentzian metric.
\een
\ee