\section{Lie Groups}

\bd [Lie Group]
A \textbf{Lie group}\index{Lie group} is a group $(G,\bullet)$, where $G$ is a smooth manifold and the maps
\bi{rrCl}
\mu \cl & G\times G & \to & G\\
& (g_1,g_2) & \mapsto & g_1\bullet g_2
\ei
and
\bi{rrCl}
i \cl & G & \to & G\\
& g & \mapsto & g^{-1}
\ei
are both smooth. Note that $G\times G$ inherits a smooth atlas from the smooth atlas of $G$.
\ed

\bd [Dimension Of Lie Group]
The \textbf{dimension} of a Lie group $(G,\bullet)$ is the dimension of $G$ as a manifold.
\ed

\be
\ben[label=\alph*)]
\item Consider $(\R^n,+)$, where $\R^n$ is understood as a smooth $n$-dimensional manifold. This is a commutative (or abelian) Lie group (since $\bullet$ is commutative), often called the $n$-dimensional translation group.

\item Let $S^1:=\{z\in\C\mid |z|=1\}$ and let $\cdot$ be the usual multiplication of complex numbers. Then $(S^1,\cdot)$ is a commutative Lie group usually denoted $\mathrm{U}(1)$.

\item Let $\mathrm{GL}(n,\R)=\{\phi\cl\R^n\xrightarrow{\sim}\R^n\mid \det \phi \neq 0\}$. This set can be endowed with the structure of a smooth $n^2$-dimensional manifold, by noting that there is a bijection between linear maps $\phi\cl\R^n\xrightarrow{\sim}\R^n$ and $\R^{2n}$. The condition $\det \phi\neq 0$ is a so-called \emph{open condition}, meaning that $\mathrm{GL}(n,\R)$ can be identified with an open subset of $\R^{2n}$, from which it then inherits a smooth structure.

Then, $(\mathrm{GL}(n,\R),\circ)$ is a Lie group called the \emph{general linear group}.
\een
\ee

\bd [Lie Group Homomorphism]
Let $(G,\bullet)$ and $(H,\circ)$ be Lie groups. A map $\phi\cl G \to H$ is \textbf{Lie group homomorphism} if it is a group homomorphism and a smooth map.
\ed

\bd [Lie Group Isomorphism]
A \textbf{Lie group isomorphism}\index{isomorphism!of Lie groups} is a Lie group homomorphism which is also a diffeomorphism of the underlying manifolds. 
\ed

\subsection{The Left Translation Map}

To every element of a Lie group there is associated a special map. Note that everything we will do here can be done equivalently by using right translation maps. 

\bd [Left Translation]
Let $(G,\bullet)$ be a Lie group and let $g\in G$. The map
\bi{rrCl}
\ell_g \cl & G & \to & G\\
& h & \mapsto & \ell_g(h):=g\bullet h \equiv gh
\ei
is called the \textbf{left translation}\index{left translation} by $g$.
\ed
If there is no danger of confusion, we usually suppress the $\bullet$ notation.  
\bp
Let $G$ be a Lie group. For any $g\in G$, the left translation map $\ell_g\cl G \to G$ is a diffeomorphism.
\ep

\bq
Let $h,h'\in G$. Then, we have
\bse
\ell_g(h)=\ell_g(h')\ \Leftrightarrow\ g h = g h' \ \Leftrightarrow\ h=h'.
\ese
Moreover, for any $h\in G$, we have $g^{-1} h\in G$ and
\bse
\ell_g(g^{-1} h) = gg^{-1} h = h.
\ese
Therefore, $\ell_g$ is a bijection on $G$.\\

Note that
\bse
\ell_g = \mu(g,-)
\ese
and since $\mu\cl G\times G \to G$ is smooth by definition, so is $\ell_g$. \\

The inverse map is $(\ell_g)^{-1}=\ell_{g^{-1}}$, since
\bse
\ell_{g^{-1}} \circ \ell_{g} = \ell_{g} \circ \ell_{g^{-1}} = \id_G.
\ese
Then, for the same reason as above with $g$ replaced by $g^{-1}$, the inverse map $(\ell_g)^{-1}$ is also smooth. Hence, the map $\ell_g$ is indeed a diffeomorphism.
\eq

Note that, in general, $\ell_g$ is not an isomorphism of groups, i.e.\ 
\bse
\ell_g(hh') \neq \ell_g(h)\,\ell_g(h')
\ese
in general. However, as the final part of the previous proof suggests, we do have
\bse
\ell_g \circ \ell_h = \ell_{gh}
\ese
for all $g,h\in G$. \\

Recall from the previous chapter than once we have a diffeomorphism $\phi$ between two manifolds $M$ and $N$, we can define the push-forward of a vector field X as
\bse
(\phi_* X)|_{\phi(p)} := \phi_* (X|_p)
\ese

where $X|_p$ is the vector created by the field $X$ on point $p$. \\

Coming in our case, we just showed that the map $\ell_g\cl G\to G$ is a diffeomorphism so we can push-forward any vector field $X$ on $G$ to another vector field (again on $G$ since the maps is between the same manifold). So in our case $\phi_* (X) = (\ell_g)_* (X)$ and for any point $h$ in $G$: $\ell_g (h) = gh$ so the push-forward equation reads 
\bse
(\ell_{g*} X)|_{gh} := (\ell_g)_* (X|_h)
\ese

\subsection{The Lie Algebra Of A Lie Group}

In Lie theory, we are typically not interested in general vector fields, but rather on special class of vector fields which are invariant under the induced push-forward of the left translation maps $\ell_g$.

\bd [Left Invariant Vector Fields]
Let $G$ be a Lie group. A vector field $X\in\Gamma(TG)$ is said to be \textbf{left-invariant}\index{left-invariant vector field} if
\bse
\forall \, g \in G  : \ (l_g)_*(X) = X.
\ese
Equivalently, we can require this to hold pointwise
\bse
\forall \, g,h \in G : \ (\ell_g)_*(X|_h) = X|_{gh}.
\ese
\ed

We can manipulate a bit the pointwise formulation to yield another reformulation. Since both sides are vectors we can let them act on a function $f$ 
\bse
(\ell_g)_*(X|_h) f = X|_{gh} f
\ese

By using the definition of a push-forward of a vector $(\phi_*)_p (X) f := X(f \circ \phi)$ the left part of the equation reads
\bse
(\ell_g)_*(X|_h) f = X|_h (f \circ \ell_g) = \big(X (f \circ \ell_g) \big)|_h
\ese

\vspace{5pt}

The right part can be manipulated as follows
\bse
X|_{gh} f = (Xf)|_{gh} = \Big( (Xf) \circ  \ell_g \Big)|_h
\ese

By substituting both final expressions back to the original one and discarding the point $h$ since they must be true for any $h$ we obtain the last reformulation of the push-forward
\bse
X (f \circ \ell_g) = X(f) \circ \ell_g.
\ese

\bd [$\mathcal{L}(G)$ (As A Set)]
We denote the set of all left-invariant vector fields on $G$ as $\mathcal{L}(G)$. 
\ed

Of course,
\bse
\mathcal{L}(G)\se\Gamma(TG)
\ese

\vspace{3pt}

but, in fact, more is true. One can check that $\mathcal{L}(G)$ is closed under 
\bi{c}
+\cl \mathcal{L}(G)\times \mathcal{L}(G) \to \mathcal{L}(G)\\
\cdot  \cl \mathcal{C}^\infty(G)\times \mathcal{L}(G) \to \mathcal{L}(G),
\ei
therefore, $\mathcal{L}(G)$ is a $\mathcal{C}^\infty(G)$-submodule of $\Gamma(TG)$, but it is also an $\R$-vector subspace of $\Gamma(TG)$. \\

Recall that, up to now, we have refrained from thinking of $\Gamma(TG)$ as an $\R$-vector space since it is infinite-dimensional and, even worse, a basis is in general uncountable. A priori, this could be true for $\mathcal{L}(G)$ as well, but we will see that the situation is, in fact, much nicer as $\mathcal{L}(G)$ will turn out to be a finite-dimensional vector space over $\R$. 

\begin{theorem}
Let $G$ be a Lie group with identity element $e\in G$. Then $\mathcal{L}(G)\cong_\mathrm{vec} T_eG$.
\end{theorem}

\bq
We will construct a linear isomorphism $j\cl T_eG\xrightarrow{\sim}\mathcal{L}(G)$. Define
\bi{rrCl}
j \cl & T_eG& \to & \Gamma(TG)\\
& A & \mapsto & j(A),
\ei
where $j(A)$ is define as
\bi{rrCl}
j(A) \cl & G& \to & TG\\
& g & \mapsto & j(A)|_g := (\ell_g)_*(A).
\ei

Now we have to prove that this is actually a linear isomorphism, and we will do it in steps.

\ben[label=\roman*)]
\item First, we show that for any $A\in T_eG$, $j(A)$ is a smooth vector field on $G$. It suffices to check that for any $f\in \mathcal{C}^\infty(G)$, we have $j(A)(f)\in \mathcal{C}^\infty(G)$. Indeed
\bi{rCl}
(j(A)(f))(g) & = & j(A)|_g(f)\\
& := &  (\ell_g)_*(A)(f)\\
& = &  A(f\circ\ell_g)\\
& = &  (f\circ\ell_g\circ\gamma)'(0),
\ei
where $\gamma$ is a curve through $e\in G$ whose tangent vector at $e$ is $A$. The map
\bi{rrClCl}
\varphi \cl & \R\times G &\to & \R &&\\
&(t,g)&\mapsto & \varphi(t,g) & := &(f\circ\ell_{g}\circ\gamma)(t) \\
&&&& = & f(g\gamma(t))
\ei
is a composition of smooth maps, hence it is smooth. Then
\bse
(j(A)(f))(g) = (\partial_1\varphi)(0,g)
\ese
depends smoothly on $g$ and thus $j(A)(f)\in \mathcal{C}^\infty(G)$.
\item We proved that $j(A)$ is indeed a smooth vector field, however now we need to prove that it is a left invariant vector field since it is an element of $\Gamma(TG)$. Let $g,h\in G$. Then, for every $A\in T_eG$, we have
\bi{rCl}
(\ell_g)_*(j(A)|_h) & := & (\ell_g)_*((\ell_h)_*(A))\\
& = & (\ell_{gh})_*(A)\\
& = & j(A)|_{gh},
\ei
so $j(A)\in \mathcal{L}(G)$. Hence, the map $j$ is really $j\cl T_eG \to \mathcal{L}(G)$.
\item We also need to check the linearity. Let $A,B\in T_eG$ and $\lambda \in \R$. Then, for any $g\in G$
\bi{rCl}
j(\lambda A + B)|_g & = & (\ell_g)_* (\lambda A + B)\\
& = & \lambda (\ell_g)_*( A) +  (\ell_g)_*(B)\\
& = & \lambda j(A)|_g + j(B)|_g,
\ei
since the push-forward is an $\R$-linear map. Hence, we have $j\cl T_eG \xrightarrow{\sim} \mathcal{L}(G)$.
\item We also need to check that the map is injective. Let $A,B\in T_eG$. Then
\bi{rCl}
j(A) = j(B) & \Leftrightarrow & \forall \, g \in G : j(A)|_g = j(B)|_g\\
&\Rightarrow & j(A)|_e = j(B)|_e\\
 & \Leftrightarrow & (\ell_e)_*(A) = (\ell_e)_*(B)\\
& \Leftrightarrow & A=B,
\ei
since $(\ell_e)_*=\id_{TG}$. Hence, the map $j$ is injective.
\item Finally we nee to check that the map is surjective. Let $X\in \mathcal{L}(G)$. Define $A^X:=X|_e\in T_eG$. Then, we have
\bse
j(A^X)|_g = (\ell_g)_*(A^X)=(\ell_g)_*(X|_e) = X_{ge}=X_g,
\ese
since $X$ is left-invariant. Hence $X=j(A^X)$ and thus $j$ is surjective. 
\een
Therefore, $j\cl T_eG\xrightarrow{\sim}\mathcal{L}(G)$ is indeed a linear isomorphism.
\eq

\bc
The space $\mathcal{L}(G)$ is finite-dimensional and $\dim \mathcal{L}(G)=\dim G$.
\ec

We will soon see that the identification of $\mathcal{L}(G)$ and $T_eG$ goes beyond the level of linear isomorphism as vector spaces, as they are isomorphic as Lie algebras. Recall from the Lie algebra chapter in the notes, that a Lie algebra over an algebraic field $K$ is a vector space over $K$ equipped with a Lie bracket $[-,-]$, i.e.\ a $K$-bilinear, antisymmetric map which satisfies the Jacobi identity. \\

Given $X,Y\in \Gamma(TM)$, we defined their Lie bracket, or commutator, as
\bse
[X,Y] (f):= X(Y(f))-Y(X(f))
\ese
for any $f\in \mathcal{C}^\infty(M)$. You can check that indeed $[X,Y]\in\Gamma(TM)$, and that the bracket is $\R$-bilinear, antisymmetric and satisfies the Jacobi identity. Thus, $(\Gamma(TM),+,\cdot,[-,-])$ is an infinite-dimensional Lie algebra over $\R$. We suppress the $+$ and $\cdot$ when they are clear from the context.
In the case of a manifold that is also a Lie group, we have the following.
\begin{theorem}
Let $G$ be a Lie group. Then $\mathcal{L}(G)$ is a Lie subalgebra of $\Gamma(TG)$.
\end{theorem}
\bq
A Lie subalgebra of a Lie algebra is simply a vector subspace which is closed under the action of the Lie bracket. Therefore, we only need to check that
\bse
\forall \, X,Y \in \mathcal{L}(G) : \ [X,Y]\in \mathcal{L}(G).
\ese
Let $X,Y \in \mathcal{L}(G)$. For any $g\in G$ and $f\in\mathcal{C}^\infty(G)$, we have
\bi{rCl}
[X,Y](f\circ\ell_g) & := & X(Y(f\circ\ell_g))-Y(X(f\circ\ell_g))\\
& = & X(Y(f)\circ\ell_g)-Y(X(f)\circ\ell_g)\\
& = & X(Y(f))\circ\ell_g-Y(X(f))\circ\ell_g\\
& = & \bigl(X(Y(f))-Y(X(f))\bigr)\circ\ell_g\\
& = & [X,Y](f)\circ\ell_g.
\ei
Hence, $[X,Y]$ is left-invariant.
\eq

\bd [$\mathcal{L}(G)$ (As An Algebra)]
Let $G$ be a Lie group. The \textbf{associated Lie algebra} of $G$ is $\mathcal{L}(G)$.
\ed

Notice that we began with $\mathcal{L}(G)$ as a set of all left invariant vector fields of $G$, which is a subset of $\Gamma(TG)$, then we inherited the $+$ and $\cdot$ of $\Gamma(TG)$ to $\mathcal{L}(G)$ and we showed that it is also a submodule and a subvector space of $\Gamma(TG)$, and finally we inherited the Lie bracket from $\Gamma(TG)$ and we showed that it is also a subalgebra of $\Gamma(TG)$. From now on when we will be referring to $\mathcal{L}(G)$, we will mean its algebra structure.\\

Given the nature of $\mathcal{L}(G)$, it is a rather complicated object, since its elements are vector fields, hence we would like to work with $T_eG$ instead, whose elements are tangent vectors. We have already shown that $\mathcal{L}(G)$ and $T_eG$ are isomorphic as vector spaces, but we would like them to be also isomorphic as algebras. Indeed, we can use the bracket on $L(G)$ to define a bracket on $T_eG$ such that they be isomorphic as Lie algebras. First, let us define the isomorphism of Lie algebras.

\bd [Lie Algebra Homomorphism]
Let $(L_1,[-,-]_{L_1})$ and $(L_2,[-,-]_{L_2})$ be Lie algebras over the same field. A linear map $\phi\cl L_1 \xrightarrow{\sim}L_2$ is a \textbf{Lie algebra homomorphism} if
\bse
\forall \, x,y\in L_1 : \ \phi([x,y]_{L_1}) = [\phi(x),\phi(y)]_{L_2}.
\ese
\ed

\bd [Lie Algebra Isomorphism]
A bijective Lie algebra homomorphism, is called a \textbf{Lie algebra isomorphism}\index{isomorphism!of Lie algebras} and we write $L_1\cong_\mathrm{Lie\, alg} L_2$.
\ed

By using the bracket $[-,-]_{\mathcal{L}(G)}$ on $\mathcal{L}(G)$ we can define, for any $A,B\in T_eG$
\bse
[A,B]_{T_eG} := j^{-1} \bigl( [j(A),j(B)]_{\mathcal{L}(G)} \bigr),
\ese
where $j^{-1}(X)=X|_e$. Equipped with these brackets, we have
\bse
\mathcal{L}(G)\cong_\mathrm{Lie\, alg}T_eG.
\ese

Hence, given a Lie group we have seen how we can construct a Lie algebra as the space of left-invariant vector fields and this algebra is isomorphic to the algebra of tangent vectors at the identity. We will later explore the opposite direction, i.e.\ given a Lie algebra, we will see how to construct a Lie group whose associated Lie algebra is the one we started from. 

\section{Application - Part 2: $\SL(2,\C)$}

In the first part of of the application in the previous chapter, we defined the set $\SL(2,\C)$ as a subset of $\C^4:=\C\times\C\times\C\times\C$. Then we showed that:
\bit
\item $\SL(2,\C)$ can be made into a group
\item $\SL(2,\C)$ can be made into a topological space
\item $\SL(2,\C)$ can be made into a topological manifold
\item $\SL(2,\C)$ can be made into a complex differentiable manifold
\eit

Hence we have left with  $\SL(2,\C)$ as a 3-dimensional, complex differentiable manifold.

\subsubsection{$\SL(2,\C)$ As A Lie Group}

We equipped $\SL(2,\C)$ with both a group and a manifold structure. In order to obtain a Lie group structure, we have to check that these two structures are compatible, that is, we have to show that the two maps
\bi{rrCl}
\mu \cl & \SL(2,\C) \times \SL(2,\C) & \to & \SL(2,\C)\\[3pt]
& (\biggl(\begin{matrix} a & b \\ c & d\end{matrix}\biggr) ,\biggl(\begin{matrix} e & f \\ g & h\end{matrix}\biggr))  & \mapsto & \biggl(\begin{matrix} a & b \\ c & d\end{matrix}\biggr) \bullet \biggl(\begin{matrix} e & f \\ g & h\end{matrix}\biggr)
\ei
and 
\bi{rrCl}
i \cl & \SL(2,\C) & \to & \SL(2,\C)\\[3pt]
& \biggl(\begin{matrix} a & b \\ c & d\end{matrix}\biggr)  & \mapsto &\biggl(\begin{matrix} a & b \\ c & d\end{matrix}\biggr)^{-1} %= \biggl(\begin{matrix} d & -b \\ -c & a\end{matrix}\biggr)
\ei
are differentiable with respect to the differentiable structure on $\SL(2,\C)$. For instance, for the inverse map $i$, we have to show that the map $y\circ i \circ x^{-1}$ is differentiable in the usual for any pair of charts $(U,x),(V,y)\in \mathscr{A}$. 

\bse
\begin{tikzcd}
U \se\SL(2,\C) \ar[dd,"x"]\ar[rr,"i"]&& V\se \SL(2,\C)\ar[dd,"y"]\\
&&\\
x(U) \se \C^3 \ar[rr,"y\circ i\circ x^{-1}"]&& y(V)\se \C^3
\end{tikzcd}
\ese

\vspace{10pt}

However, since $\SL(2,\C)$ is connected, the differentiability of the transition maps in $\mathscr{A}$ implies that if $y\circ i\circ x^{-1}$ is differentiable for any two given charts, then it is differentiable for all charts in $\mathscr{A}$. Hence, we can simply let $(U,x)$ and $(V,y)$ be the two charts on $\SL(2,\C)$ defined above. Then, we have
\bse
(y\circ i\circ x^{-1}) (a,b,c) = (y\circ i) ( \biggl(\begin{matrix} a & b \\ c & \frac{1+bc}{a}\end{matrix}\biggr) ) = y ( \biggl(\begin{matrix} \frac{1+bc}{a} & -b \\ -c & a\end{matrix}\biggr)) = (\tfrac{1+bc}{a},-b,a)
\ese
which is certainly complex differentiable as a map between open subsets of $\C^3$ (recall that $a\neq 0$ on $x(U)$). \\

Checking that $\mu$ is complex differentiable is slightly more involved, since we first have to equip $\SL(2,\C) \times \SL(2,\C)$ with a suitable ``product differentiable structure'' and then proceed as above. Once that is done, we can finally conclude that $((\SL(2,\C),\mathcal{O},\mathscr{A}),\bullet)$ is a $3$-dimensional complex Lie group.

\subsection{The Lie Algebra Of \texorpdfstring{$\SL(2,\C)$}{SL(2,C)}}

Recall that to every Lie group $G$, there is an associated Lie algebra $\mathcal{L}(G)$, where
\bse
\mathcal{L}(G) := \{X\in \Gamma(TG)\mid \forall \, g,h\in G : (\ell_g)_*(X|_h)=X_{gh}\},
\ese
which we then proved to be isomorphic to the Lie algebra $T_eG$ with Lie bracket
\bse
[A,B]_{T_eG} := j^{-1} ([j(A),j(B)]_{\mathcal{L}(G)})
\ese
induced by the Lie bracket on $\mathcal{L}(G)$ via the isomorphism $j$
\bse
j(A)|_g := (\ell_g)_*(A).
\ese
In the case of $\SL(2,\C)$, the left translation map by $\left(\begin{smallmatrix}a & b \\ c & d\end{smallmatrix}\right)$ is 
\bi{rrCl}
\ell_{\left(\begin{smallmatrix}a & b \\ c & d\end{smallmatrix}\right)} \cl & \SL(2,\C) &\to & \SL(2,\C)\\
& \biggl(\begin{matrix}e & f \\ g  & h\end{matrix}\biggr) & \mapsto & \biggl(\begin{matrix}a & b \\ c & d\end{matrix}\biggr) \bullet \biggl(\begin{matrix}e & f \\ g  & h\end{matrix}\biggr) 
\ei
By using the standard notation $\sl(2,\C)\equiv \mathcal{L}(\SL(2,\C))$, we have
\bse
\sl(2,\C) \cong_{\mathrm{Lie \, alg}} T_{\left(\begin{smallmatrix}1 & 0 \\ 0 & 1\end{smallmatrix}\right)}\SL(2,\C).
\ese
We would now like to explicitly determine the Lie bracket on $T_{\left(\begin{smallmatrix}1 & 0 \\ 0 & 1\end{smallmatrix}\right)}\SL(2,\C)$, and hence determine its structure constants.

Recall that if $(U,x)$ is a chart on a manifold $M$ and $p\in U$, then the chart $(U,x)$ induces a basis of the tangent space $T_pM$. We shall use our previously defined chart $(U,x)$ on $\SL(2,\C)$, where $U:= \{ \left( \begin{smallmatrix} a & b \\ c & d \end{smallmatrix}\right) \in \SL(2,\C) \mid a \neq 0 \}$ and 
\bi{rrCl}
x \cl & U & \to & x(U) \se \C^3\\
& \biggl( \begin{matrix} a & b \\ c & d \end{matrix}\biggr) & \mapsto & (a,b,c).
\ei
Note that the $d$ appearing here is completely redundant, since the membership condition of $\SL(2,\C)$ forces $d=\frac{1+bc}{a}$. However, we will keep writing the $d$ to avoid having a fraction in a matrix in a subscript.

The chart $(U,x)$ contains $\left(\begin{smallmatrix}1 & 0 \\ 0 & 1\end{smallmatrix}\right)$ and hence we get an induced co-ordinate basis
\bse
\biggl\{\tvb{x}{i}{\left(\begin{smallmatrix}1 & 0 \\ 0 & 1\end{smallmatrix}\right)}\in  T_{\left(\begin{smallmatrix}1 & 0 \\ 0 & 1\end{smallmatrix}\right)}\SL(2,\C) \ \Big| \ 1\leq i \leq 3 \biggr\}
\ese
so that any $A\in  T_{\left(\begin{smallmatrix}1 & 0 \\ 0 & 1\end{smallmatrix}\right)}\SL(2,\C)$ can be written as
\bse
A = \alpha \tvb{x}{1}{\left(\begin{smallmatrix}1 & 0 \\ 0 & 1\end{smallmatrix}\right)} + \beta \tvb{x}{2}{\left(\begin{smallmatrix}1 & 0 \\ 0 & 1\end{smallmatrix}\right)} + \gamma \tvb{x}{3}{\left(\begin{smallmatrix}1 & 0 \\ 0 & 1\end{smallmatrix}\right)},
\ese
for some $\alpha,\beta,\gamma\in \C$. Since the Lie bracket is bilinear, its action on these basis vectors uniquely extends to the whole of $T_{\left(\begin{smallmatrix}1 & 0 \\ 0 & 1\end{smallmatrix}\right)}\SL(2,\C)$ by linear continuation. Hence, we simply have to determine the action of the Lie bracket of $\sl(2,\C)$ on the images under the isomorphism $j$ of these basis vectors. 

Let us now determine the image of these co-ordinate induced basis elements under the isomorphism $j$. The object
\bse
j\biggl(    \tvb{x}{i}{\left(\begin{smallmatrix}1 & 0 \\ 0 & 1\end{smallmatrix}\right)}\biggr) \in \sl(2,\C) 
\ese
is a left-invariant vector field on $\SL(2,\C)$. It assigns to each point $\left(\begin{smallmatrix}a & b \\ c & d\end{smallmatrix}\right)\in U\se \SL(2,\C)$ the tangent vector
\bse
j\biggl(    \tvb{x}{i}{\left(\begin{smallmatrix}1 & 0 \\ 0 & 1\end{smallmatrix}\right)}\biggr) \bigg|_{\left(\begin{smallmatrix}a & b \\ c & d\end{smallmatrix}\right)} : =
\Bigl(\ell_{\left(\begin{smallmatrix}a & b \\ c & d\end{smallmatrix}\right)} \Bigr)_*     \tvb{x}{i}{\left(\begin{smallmatrix}1 & 0 \\ 0 & 1\end{smallmatrix}\right)} \in T_{\left(\begin{smallmatrix}a & b \\ c & d\end{smallmatrix}\right)}\SL(2,\C). 
\ese
This tangent vector is a $\C$-linear map $\mathcal{C}^\infty(\SL(2,\C))\xrightarrow{\sim}\C$, where $\mathcal{C}^\infty(\SL(2,\C))$ is the $\C$-vector space (in fact, the $\C$-algebra) of smooth complex-valued functions on $\SL(2,\C)$ although, to be precise, since we are working in a chart we should only consider functions defined on $U$. For (the restriction to $U$ of) any $f\in \mathcal{C}^\infty(\SL(2,\C))$ we have, explicitly,
\bi{rCl}
\Bigl(\ell_{\left(\begin{smallmatrix}a & b \\ c & d\end{smallmatrix}\right)} \Bigr)_*     \tvb{x}{i}{\left(\begin{smallmatrix}1 & 0 \\ 0 & 1\end{smallmatrix}\right)} (f) & = & \tvb{x}{i}{\left(\begin{smallmatrix}1 & 0 \\ 0 & 1\end{smallmatrix}\right)} \Bigl(f\circ \ell_{\left(\begin{smallmatrix}a & b \\ c & d\end{smallmatrix}\right)} \Bigr)\\
& = & \partial_i\Bigl(f\circ \ell_{\left(\begin{smallmatrix}a & b \\ c & d\end{smallmatrix}\right)} \circ x^{-1}\Bigr) (x\left(\begin{smallmatrix}1 & 0 \\ 0 & 1\end{smallmatrix}\right)),
\ei
where the argument of $\partial_i$ in the last line is a map $x(U)\se\C^3\to\C$, hence $\partial_i$ is simply the operation of complex differentiation with respect to the $i$-th (out of the 3) complex variable of the map $f\circ \ell_{\left(\begin{smallmatrix}a & b \\ c & d\end{smallmatrix}\right)} \circ x^{-1}$, which is then to be evaluated at $x\left(\begin{smallmatrix}1 & 0 \\ 0 & 1\end{smallmatrix}\right)\in \C^3$. By inserting an identity in the composition, we have
\bi{rCl}
& = &\partial_i\Bigl(f\circ {\id_U} \circ \ell_{\left(\begin{smallmatrix}a & b \\ c & d\end{smallmatrix}\right)} \circ x^{-1}\Bigr) (x\left(\begin{smallmatrix}1 & 0 \\ 0 & 1\end{smallmatrix}\right)) \\
& = & \partial_i\Bigl(f\circ ( x^{-1}\circ x) \circ \ell_{\left(\begin{smallmatrix}a & b \\ c & d\end{smallmatrix}\right)} \circ x^{-1}\Bigr) (x\left(\begin{smallmatrix}1 & 0 \\ 0 & 1\end{smallmatrix}\right))\\
& = & \partial_i\Bigl((f\circ  x^{-1})\circ (x \circ \ell_{\left(\begin{smallmatrix}a & b \\ c & d\end{smallmatrix}\right)} \circ x^{-1})\Bigr) (x\left(\begin{smallmatrix}1 & 0 \\ 0 & 1\end{smallmatrix}\right)),
\ei
where $f\circ  x^{-1}\cl x(U)\se \C^3 \to \C$ and $(x \circ \ell_{\left(\begin{smallmatrix}a & b \\ c & d\end{smallmatrix}\right)} \circ x^{-1})\cl x(U)\se \C^3 \to x(U)\se\C^3$ and hence, we can use the multi-dimensional chain rule to obtain
\bi{rCl}
& = & \Bigl(\partial_m(f\circ  x^{-1})\bigl((x \circ \ell_{\left(\begin{smallmatrix}a & b \\ c & d\end{smallmatrix}\right)} \circ x^{-1}) (x\left(\begin{smallmatrix}1 & 0 \\ 0 & 1\end{smallmatrix}\right))\bigr)\Bigr)\Bigl( 
\partial_i (x^m \circ \ell_{\left(\begin{smallmatrix}a & b \\ c & d\end{smallmatrix}\right)} \circ x^{-1}) (x\left(\begin{smallmatrix}1 & 0 \\ 0 & 1\end{smallmatrix}\right))\Bigr),
\ei
with the summation going from $m=1$ to $m=3$. The first factor is simply
\bi{rCl}
\partial_m(f\circ  x^{-1})\bigl((x \circ \ell_{\left(\begin{smallmatrix}a & b \\ c & d\end{smallmatrix}\right)}) \left(\begin{smallmatrix}1 & 0 \\ 0 & 1\end{smallmatrix}\right)\bigr) & =\phantom{:} & \partial_m(f\circ  x^{-1})(x\left(\begin{smallmatrix}a & b \\ c & d\end{smallmatrix}\right))\\
& =: & \tvb{x}{m}{\left(\begin{smallmatrix}a & b \\ c & d\end{smallmatrix}\right)} (f) .
\ei
To see what the second factor is, we first consider the map $x^m \circ \ell_{\left(\begin{smallmatrix}a & b \\ c & d\end{smallmatrix}\right)} \circ x^{-1}$. This map acts on the triple $(e,f,g)\in x(U)$ as
\bi{rCl}
(x^m \circ \ell_{\left(\begin{smallmatrix}a & b \\ c & d\end{smallmatrix}\right)} \circ x^{-1}) (e,f,g) & = & (x^m \circ \ell_{\left(\begin{smallmatrix}a & b \\ c & d\end{smallmatrix}\right)} ) \biggl(\begin{matrix}e & f \\ g & \frac{1+fg}{e}\end{matrix}\biggr)\\
& = & x^m (\biggl(\begin{matrix}a & b \\ c & d\end{matrix}\biggr) \bullet \biggl(\begin{matrix}e & f \\ g & \frac{1+fg}{e}\end{matrix}\biggr))\\
& = & x^m (\left(\begin{matrix}ae+bg &\, af+ \frac{b(1+fg)}{e} \\ ce+dg &\, cf+\frac{d(1+fg)}{e}\end{matrix}\right) ),
\ei
and since $x^m := {\proj_m} \circ x$, with $m\in \{1,2,3\}$, we have 
\bse
(x^m \circ \ell_{\left(\begin{smallmatrix}a & b \\ c & d\end{smallmatrix}\right)} \circ x^{-1}) (e,f,g) = \proj_m (ae+bg, af+ \tfrac{b(1+fg)}{e}, ce+dg ),
\ese
the map $\proj_m$ simply picks the $m$-th component of the triple. We now have to apply $\partial_i$ to this map, with $i\in \{1,2,3\}$, i.e.\ we have to differentiate with respect to each of the three complex variables $e$, $f$, and $g$. We can write the result as
\bse
\partial_i(x^m \circ \ell_{\left(\begin{smallmatrix}a & b \\ c & d\end{smallmatrix}\right)} \circ x^{-1}) (e,f,g)= D(e,f,g)^m_{\phantom{m}i},
\ese
where $m$ labels the rows and $i$ the columns of the matrix
\bse
D(e,f,g)= \left(\begin{matrix}a & 0 & b\\ -\frac{b(1+fg)}{e^2} &\,a+\frac{bg}{e} &\frac{bf}{e}\\ c & 0 & d\end{matrix}\right).
\ese
Finally, by evaluating this at $(e,f,g)=x\left(\begin{smallmatrix}1 & 0 \\ 0 & 1\end{smallmatrix}\right) = (1,0,0)$, we obtain
\bse
\partial_i(x^m \circ \ell_{\left(\begin{smallmatrix}a & b \\ c & d\end{smallmatrix}\right)} \circ x^{-1}) (x\left(\begin{smallmatrix}1 & 0 \\ 0 & 1\end{smallmatrix}\right))= D^m_{\phantom{m}i},
\ese
where, by recalling that $d=\frac{1+bc}{a}$,
\bse
D:= D(1,0,0)= \left(\begin{matrix}a & 0 & b\ \\ -b & a & 0\\ c & 0 & \frac{1+bc}{a}\end{matrix}\right).
\ese
Putting the two factors back together yields
\bse
\Bigl(\ell_{\left(\begin{smallmatrix}a & b \\ c & d\end{smallmatrix}\right)} \Bigr)_*     \tvb{x}{i}{\left(\begin{smallmatrix}1 & 0 \\ 0 & 1\end{smallmatrix}\right)} (f) =  D^m_{\phantom{m}i} \tvb{x}{m}{\left(\begin{smallmatrix}a & b \\ c & d\end{smallmatrix}\right)} (f) .
\ese
Since this holds for an arbitrary $f\in\mathcal{C}^\infty(\SL(2,\C))$, we have
\bse
j\biggl(    \tvb{x}{i}{\left(\begin{smallmatrix}1 & 0 \\ 0 & 1\end{smallmatrix}\right)}\biggr) \bigg|_{\left(\begin{smallmatrix}a & b \\ c & d\end{smallmatrix}\right)} : =
\Bigl(\ell_{\left(\begin{smallmatrix}a & b \\ c & d\end{smallmatrix}\right)} \Bigr)_*     \tvb{x}{i}{\left(\begin{smallmatrix}1 & 0 \\ 0 & 1\end{smallmatrix}\right)} = D^m_{\phantom{m}i} \tvb{x}{m}{\left(\begin{smallmatrix}a & b \\ c & d\end{smallmatrix}\right)}, 
\ese
and since the point $\left(\begin{smallmatrix}a & b \\ c & d\end{smallmatrix}\right)\in U\se\SL(2,\C)$ is also arbitrary, we have
\bse
j\biggl( \tvb{x}{i}{\left(\begin{smallmatrix}1 & 0 \\ 0 & 1\end{smallmatrix}\right)}\biggr) = D^m_{\phantom{m}i}\, \frac{\partial}{\partial x^m} \in \sl(2,\C),
\ese
where $D$ is now the corresponding matrix of co-ordinate functions
\bse
D:=\left(\begin{matrix}x^1 & 0 & x^2\ \\ -x^2 & x^1 & 0\\ x^3 & 0 & \frac{1+x^2x^3}{x^1}\end{matrix}\right).
\ese
Note that while the three vector fields
\bi{rrCl}
\frac{\partial}{\partial x^m} \cl &  \SL(2,\C) & \to &  T\SL(2,\C)\\
& \biggl(\begin{matrix}a & b \\ c & d\end{matrix}\biggr) & \mapsto & \tvb{x}{m}{\left(\begin{smallmatrix}a & b \\ c & d\end{smallmatrix}\right)}
\ei
are not individually left-invariant, their linear combination with coefficients $D^m_{\phantom{m}i}$ is indeed left-invariant. Recall that these vector fields
\ben[label=\roman*)]
\item are $\C$-linear maps
\bi{rrCl}
\frac{\partial}{\partial x^m}\cl &\mathcal{C}^\infty(\SL(2,\C))&\xrightarrow{\sim} &\mathcal{C}^\infty(\SL(2,\C))\\
& f & \mapsto & \partial_m(f\circ x^{-1})\circ x;
\ei
\item satisfy the Leibniz rule
\bse
\frac{\partial}{\partial x^m} (fg) = f\frac{\partial}{\partial x^m}(g)+g\frac{\partial}{\partial x^m}(f);
\ese
\item act on the coordinate functions $x^i\in \mathcal{C}^\infty(\SL(2,\C))$ as
\bse
\frac{\partial}{\partial x^m} (x^i) = \partial_m (x^i \circ x^{-1})\circ x = \partial_m ({\proj_i}\circ x \circ x^{-1}) \circ x =\delta^i_m\circ x= \delta^i_m,
\ese
since the composition of a constant function with any composable function is just the constant function.
\een
Hence, we have an expansion of the images of the basis of $T_{\left(\begin{smallmatrix}1 & 0 \\ 0 & 1\end{smallmatrix}\right)}\SL(2,\C)$ under $j$:
\bi{rCl}
j\biggl( \tvb{x}{1}{\left(\begin{smallmatrix}1 & 0 \\ 0 & 1\end{smallmatrix}\right)}\biggr) & = & x^1  \frac{\partial}{\partial x^1} - x^2\frac{\partial}{\partial x^2}  + x^3\frac{\partial}{\partial x^3} \\
j\biggl( \tvb{x}{2}{\left(\begin{smallmatrix}1 & 0 \\ 0 & 1\end{smallmatrix}\right)}\biggr) & = & x^1 \frac{\partial}{\partial x^2}  \\
j\biggl( \tvb{x}{3}{\left(\begin{smallmatrix}1 & 0 \\ 0 & 1\end{smallmatrix}\right)}\biggr) & = & x^2\, \frac{\partial}{\partial x^1} + \tfrac{1+x^2x^3}{x^1} \frac{\partial}{\partial x^3}  .
\ei
We now have to calculate the bracket (in $\sl(2,\C)$) of every pair of these. We can also do them all at once, which is a good exercise in index gymnastics. We have
\bse
\biggl[j\biggl( \tvb{x}{i}{\left(\begin{smallmatrix}1 & 0 \\ 0 & 1\end{smallmatrix}\right)}\biggr) ,j\biggl( \tvb{x}{k}{\left(\begin{smallmatrix}1 & 0 \\ 0 & 1\end{smallmatrix}\right)}\biggr) \biggr] =  \left[  D^m_{\phantom{m}i}\, \frac{\partial}{\partial x^m}, D^n_{\phantom{n}k}\, \frac{\partial}{\partial x^n}\right].
\ese
Letting this act on an arbitrary $f\in \mathcal{C}^\infty(\SL(2,\C))$, by definition
\bse
\left[  D^m_{\phantom{m}i}\, \frac{\partial}{\partial x^m}, D^n_{\phantom{n}k}\, \frac{\partial}{\partial x^n}\right](f) :=  D^m_{\phantom{m}i}\, \frac{\partial}{\partial x^m} \Bigl( D^n_{\phantom{n}k}\, \frac{\partial}{\partial x^n} (f)\Bigr) -  D^n_{\phantom{n}k}\, \frac{\partial}{\partial x^n} \Bigl(D^m_{\phantom{m}i}\, \frac{\partial}{\partial x^m}(f)\Bigr).
\ese
The first term gives
\bi{rCl}
 D^m_{\phantom{m}i}\, \frac{\partial}{\partial x^m} \Bigl( D^n_{\phantom{n}k}\, \frac{\partial}{\partial x^n} (f)\Bigr) & = &  D^m_{\phantom{m}i}\, \frac{\partial}{\partial x^m} ( D^n_{\phantom{n}k}\, \partial_n (f\circ x^{-1})\circ x)\\
& = & D^m_{\phantom{m}i}\, \frac{\partial}{\partial x^m} (D^n_{\phantom{n}k})\,(\partial_n (f\circ x^{-1})\circ x) +  D^m_{\phantom{m}i}D^n_{\phantom{n}k} \,\frac{\partial}{\partial x^m} (\partial_n (f\circ x^{-1})\circ x)\\
& = & D^m_{\phantom{m}i}\, \frac{\partial}{\partial x^m} (D^n_{\phantom{n}k})\,(\partial_n (f\circ x^{-1})\circ x) +  D^m_{\phantom{m}i}D^n_{\phantom{n}k} \,\partial_m(\partial_n (f\circ x^{-1})\circ x\circ x^{-1})\circ x\\
& = & D^m_{\phantom{m}i}\, \frac{\partial}{\partial x^m} (D^n_{\phantom{n}k})\,(\partial_n (f\circ x^{-1})\circ x) +  D^m_{\phantom{m}i}D^n_{\phantom{n}k} \,\partial_m\partial_n (f\circ x^{-1})\circ x.
\ei
Similarly, we have
\bse
D^n_{\phantom{n}k}\, \frac{\partial}{\partial x^n} \Bigl(  D^m_{\phantom{m}i}\, \frac{\partial}{\partial x^m} (f)\Bigr) = D^n_{\phantom{n}k}\, \frac{\partial}{\partial x^n} (D^m_{\phantom{m}i})\,(\partial_m (f\circ x^{-1})\circ x) +  D^n_{\phantom{n}k}D^m_{\phantom{m}i}\,\partial_n\partial_m (f\circ x^{-1})\circ x.
\ese
Hence, recalling that $\partial_m\partial_n=\partial_n\partial_m$ by Schwarz's theorem, we have  
\bi{rCl}
\left[  D^m_{\phantom{m}i}\, \frac{\partial}{\partial x^m}, D^n_{\phantom{n}k}\, \frac{\partial}{\partial x^n}\right](f) &=&  D^m_{\phantom{m}i}\, \frac{\partial}{\partial x^m} (D^n_{\phantom{n}k})\, (\partial_n (f\circ x^{-1})\circ x) +  \cancel[gray]{D^m_{\phantom{m}i}D^n_{\phantom{n}k} \,\partial_m\partial_n (f\circ x^{-1})\circ x}\\
& & \negmedspace {} - D^n_{\phantom{n}k}\, \frac{\partial}{\partial x^n} (D^m_{\phantom{m}i})\,(\partial_m (f\circ x^{-1})\circ x) - \cancel[gray]{D^n_{\phantom{n}k}D^m_{\phantom{m}i}\,\partial_n\partial_m (f\circ x^{-1})\circ x}\\
& = & \Bigl( D^m_{\phantom{m}i}\, \frac{\partial}{\partial x^m} (D^n_{\phantom{n}k}) - D^m_{\phantom{m}k}\, \frac{\partial}{\partial x^m} (D^n_{\phantom{n}i})\Bigr)\partial_n (f\circ x^{-1})\circ x\\
& = & \Bigl( D^m_{\phantom{m}i}\, \frac{\partial}{\partial x^m} (D^n_{\phantom{n}k}) - D^m_{\phantom{m}k}\, \frac{\partial}{\partial x^m} (D^n_{\phantom{n}i})\Bigr)\frac{\partial}{\partial x^n} (f),
\ei
where we relabelled some dummy indices. Since the $f\in\mathcal{C}^\infty(\SL(2,\C))$ was arbitrary,
\bse
\left[  D^m_{\phantom{m}i}\, \frac{\partial}{\partial x^m}, D^n_{\phantom{n}k}\, \frac{\partial}{\partial x^n}\right] =  \Bigl( D^m_{\phantom{m}i}\, \frac{\partial}{\partial x^m} (D^n_{\phantom{n}k}) - D^m_{\phantom{m}k}\, \frac{\partial}{\partial x^m} (D^n_{\phantom{n}i})\Bigr)\frac{\partial}{\partial x^n} .
\ese
We can now evaluate this explicitly. For $i=1$ and $k=2$, we have
\bi{rCl}
\left[  D^m_{\phantom{m}1} \frac{\partial}{\partial x^m}, D^n_{\phantom{n}2} \frac{\partial}{\partial x^n}\right] &=&  \Bigl( \cancel[gray]{D^m_{\phantom{m}1} \frac{\partial}{\partial x^m} (D^1_{\phantom{1}2})} - D^m_{\phantom{m}2} \frac{\partial}{\partial x^m} (D^1_{\phantom{1}1})\Bigr)\frac{\partial}{\partial x^1}\\
& &\negmedspace{}+  \Bigl( D^m_{\phantom{m}1} \frac{\partial}{\partial x^m} (D^2_{\phantom{2}2}) - D^m_{\phantom{m}2} \frac{\partial}{\partial x^m} (D^2_{\phantom{2}1})\Bigr)\frac{\partial}{\partial x^2}\\
& & \negmedspace{}+ \Bigl( \cancel[gray]{D^m_{\phantom{m}1} \frac{\partial}{\partial x^m} (D^3_{\phantom{3}2})} - D^m_{\phantom{m}2} \frac{\partial}{\partial x^m} (D^3_{\phantom{3}1})\Bigr)\frac{\partial}{\partial x^3}\\
& = & -D^1_{\phantom{1}2}\frac{\partial}{\partial x^1}+(D^1_{\phantom{1}1}+D^2_{\phantom{2}2})\frac{\partial}{\partial x^2}-D^3_{\phantom{3}2}\frac{\partial}{\partial x^3}\\
& = & 2x^1 \frac{\partial}{\partial x^2}.
\ei
Similarly, we compute
\bi{rCl}
\left[  D^m_{\phantom{m}1} \frac{\partial}{\partial x^m}, D^n_{\phantom{n}3} \frac{\partial}{\partial x^n}\right] &=&  \Bigl( D^m_{\phantom{m}1} \frac{\partial}{\partial x^m} (D^1_{\phantom{1}3}) - D^m_{\phantom{m}3} \frac{\partial}{\partial x^m} (D^1_{\phantom{1}1})\Bigr)\frac{\partial}{\partial x^1}\\
& &\negmedspace{}+  \Bigl( \cancel[gray]{D^m_{\phantom{m}1} \frac{\partial}{\partial x^m} (D^2_{\phantom{2}3})} - D^m_{\phantom{m}3} \frac{\partial}{\partial x^m} (D^2_{\phantom{2}1})\Bigr)\frac{\partial}{\partial x^2}\\
& & \negmedspace{}+ \Bigl( D^m_{\phantom{m}1} \frac{\partial}{\partial x^m} (D^3_{\phantom{3}3}) - D^m_{\phantom{m}3} \frac{\partial}{\partial x^m} (D^3_{\phantom{3}1})\Bigr)\frac{\partial}{\partial x^3}\\
& = & -2x^2\frac{\partial}{\partial x^1}-2(\tfrac{1+x^2x^3}{x^1})\frac{\partial}{\partial x^3}
\ei
and
\bi{rCl}
\left[  D^m_{\phantom{m}2} \frac{\partial}{\partial x^m}, D^n_{\phantom{n}3} \frac{\partial}{\partial x^n}\right] &=&  \Bigl( D^m_{\phantom{m}2} \frac{\partial}{\partial x^m} (D^1_{\phantom{1}3}) - \cancel[gray]{D^m_{\phantom{m}3} \frac{\partial}{\partial x^m} (D^1_{\phantom{1}2})}\Bigr)\frac{\partial}{\partial x^1}\\
& &\negmedspace{}+  \Bigl( \cancel[gray]{D^m_{\phantom{m}2} \frac{\partial}{\partial x^m} (D^2_{\phantom{2}3})} - D^m_{\phantom{m}3} \frac{\partial}{\partial x^m} (D^2_{\phantom{2}2})\Bigr)\frac{\partial}{\partial x^2}\\
& & \negmedspace{}+ \Bigl( D^m_{\phantom{m}2} \frac{\partial}{\partial x^m} (D^3_{\phantom{3}3}) - \cancel[gray]{D^m_{\phantom{m}3} \frac{\partial}{\partial x^m} (D^3_{\phantom{3}2})}\Bigr)\frac{\partial}{\partial x^3}\\
& = & (D^2_{\phantom{2}1}-D^1_{\phantom{1}3})\frac{\partial}{\partial x^1}+D^2_{\phantom{2}3}\frac{\partial}{\partial x^2}-D^3_{\phantom{3}2}\frac{\partial}{\partial x^3}\\
& = & x^1 \frac{\partial}{\partial x^1}- x^2\frac{\partial}{\partial x^2} + x^3\frac{\partial}{\partial x^3},
\ei
where the differentiation rules that we have used come from the definition of the vector field $\frac{\partial}{\partial x^m}$, the Leibniz rule, and the action on co-ordinate functions.

By applying $j^{-1}$, which is just evaluation at the identity, to these vector fields, we finally see that the induced Lie bracket on $T_{\left(\begin{smallmatrix}1 & 0 \\ 0 & 1\end{smallmatrix}\right)}\SL(2,\C)$ satisfies

\bi{rCl}
\biggl[\tvb{x}{1}{\left(\begin{smallmatrix}1 & 0 \\ 0 & 1\end{smallmatrix}\right)},\tvb{x}{2}{\left(\begin{smallmatrix}1 & 0 \\ 0 & 1\end{smallmatrix}\right)} \biggr] & = & 2\tvb{x}{2}{\left(\begin{smallmatrix}1 & 0 \\ 0 & 1\end{smallmatrix}\right)}\\[4pt]
\biggl[\tvb{x}{1}{\left(\begin{smallmatrix}1 & 0 \\ 0 & 1\end{smallmatrix}\right)},\tvb{x}{3}{\left(\begin{smallmatrix}1 & 0 \\ 0 & 1\end{smallmatrix}\right)} \biggr] & = & -2\tvb{x}{3}{\left(\begin{smallmatrix}1 & 0 \\ 0 & 1\end{smallmatrix}\right)}\\[4pt]
\biggl[\tvb{x}{2}{\left(\begin{smallmatrix}1 & 0 \\ 0 & 1\end{smallmatrix}\right)},\tvb{x}{3}{\left(\begin{smallmatrix}1 & 0 \\ 0 & 1\end{smallmatrix}\right)} \biggr] & = & \tvb{x}{1}{\left(\begin{smallmatrix}1 & 0 \\ 0 & 1\end{smallmatrix}\right)}.
\ei
Hence, the structure constants of $T_{\left(\begin{smallmatrix}1 & 0 \\ 0 & 1\end{smallmatrix}\right)}\SL(2,\C)$ with respect to the co-ordinate basis are
\bse
C^2_{\phantom{2}12} = 2, \qquad C^3_{\phantom{3}13}=-2,\qquad C^1_{\phantom{1}23}=1,
\ese
with all other being either zero or related to these by anti-symmetry.