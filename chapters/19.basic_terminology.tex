\bd[Mutual Fund]
A\textbf{mutual fund} is a type of financial vehicle made up of a pool of money collected from many investors to invest in securities like stocks, bonds, money market instruments, and other assets. 
\ed

Mutual funds are operated by professional money managers, who allocate the fund's assets and attempt to produce capital gains or income for the fund's investors.  A mutual fund's portfolio is structured and maintained to match the investment objectives stated in its prospectus.  Mutual funds give small or individual investors access to professionally managed portfolios of equities, bonds, and other securities. Each shareholder, therefore, participates proportionally in the gains or losses of the fund. Mutual funds invest in a vast number of securities, and performance is usually tracked as the change in the total market cap of the fund-derived by the aggregating performance of the underlying investments.

\bd [Equity Funds]
The largest category of mutual funds is the \textbf{equity funds} (or stock funds).  As the name implies, this sort of fund invests principally in stocks.
\ed

Before we move on to index funds we need to define what an index is

\bd[Index]
An \textbf{index} is a method to track the performance of a group of assets in a standardized way.
\ed

Indexes typically measure the performance of a basket of securities intended to replicate a certain area of the market. These could be a broad-based index that captures the entire market, such as the Standard \& Poor's 500 Index (S\&P500) or Dow Jones Industrial Average (DJIA), or more specialized such as indexes that track a particular industry or segment. Indexes are also created to measure other financial or economic data such as interest rates, inflation, or manufacturing output. Indexes often serve as benchmarks against which to evaluate the performance of a portfolio's returns.  \v

Now let's see some of the most popular indexes.

The S\&P 500,[2] or simply the S\&P,[4] is a stock market index that measures the stock performance of 500 large companies listed on stock exchanges in the United States. It is one of the most commonly followed equity indices.[5]

\bd[Indexing]
One popular investment strategy, known as indexing, is to try to replicate such an index in a passive manner rather than trying to outperform it.
\ed

Now we are in a position to define index funds.

\bd[Index Fund]
An \textbf{index fund} is a type of mutual fund with a portfolio constructed to match or track the components of a financial market index
\ed

For example an index fund could track the Standard \& Poor's 500 Index (S\&P 500).  An index mutual fund is said to provide broad market exposure,  low operating expenses, and low portfolio turnover. These funds follow their benchmark index regardless of the state of the markets.  \v

Index funds are generally considered ideal core portfolio holdings for retirement accounts, such as individual retirement accounts (IRAs) and 401(k) accounts.  Legendary investor Warren Buffett has recommended index funds as a haven for savings for the later years of life.  Rather than picking out individual stocks for investment,  he has said,  it makes more sense for the average investor to buy all of the S\&P 500 companies at the low cost an index fund offers.








