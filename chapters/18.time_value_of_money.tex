\section{Interest Rate}

\begin{definition}[Interest Rate]
\textbf{Interest rate}, or required rate of return, or discount rate, or opportunity cost $r$ is the amount charged by a lender to the borrower, expressed as a percentage of the initial loan called the principal $P$. Interest rates are usually noted on annual basis known as the annual percentage rate (APR).
\end{definition}

\begin{definition}[Risk-free Interest Rate]
The \textbf{risk-free interest rate} is the theoretical rate of return of an investment with zero risk. In practice the risk-free interest rate does not exist since even the safest investment carries a small amount of risk. Usually we define the risk-free interest rate to be the government's bonds (treasury bills).
\end{definition}

When we borrow (or lend) money from (or to) a third party we apply an interest rate for it. This interest rate splits to the following components:
\begin{itemize}
\item Risk-free Rate:\\
Because the lender could simply invest their money on bonds instead of lending them.

\item Inflation Premium:\\
Due to expected inflation.

\item Default Risk Premium:\\
Due to credit risk, i.e the risk that the borrower might not be able to pay back the loan.

\item Liquidity Premium:\\
The money borrowed are not available at any time to be reinvested.

\item Maturity Premium:\\
Depending on the time period of lending different interest rates are applied.
\end{itemize}

The risk-free rate combined together with the inflation premium are called \say{nominal risk-free rate}.\\

Based on the way we apply the interest rate on a loan we distinguish between two main categories: \say{simple interest rate} and \say{compound interest rate}.

\begin{definition}[Simple Interest Rate]
\textbf{Simple interest rate} is a fixed rate $r$ applied on the principle originally lent to the borrower:

\begin{equation}
A = (1+rt) \cdot P
\end{equation}

\vspace{10pt}

where $A$ is the money that the borrower should pay back to the lender for the loan, $r$ is the simple interest rate, $t$ is the lending period (usually expressed in years), and $P$ is the premium.
\end{definition}

\begin{definition}[Compound Interest Rate]
\textbf{Compound interest rate} is a fixed rate $r$ applied on both the principle originally lent to the borrower and the compounding interest paid on the loan:

\begin{equation}
A = (1+r)^t \cdot P
\end{equation}
\end{definition}

\vspace{10pt}

In both of these definitions the interest rate is annual and in the case of compounding interest rate the compounding interest is compounded once per year. However this in not the general case, since we can choose to compound the interest in other time periods (eg: semi-annually or daily). In order to do so we define the \say{intrayear compound interest rate}.

\begin{definition}[Intrayear Compound Interest Rate]
\textbf{Intrayear compounding interest rate} is a fixed rate $r$ applied on both the principle originally lent to the borrower and the compounding interest paid on the loan, where the compound takes place $n$ times per year:

\begin{equation}
A = \Big( 1+\frac{r}{n} \Big)^{nt} \cdot P
\end{equation}

\vspace{10pt}

where $A$ is the money that the borrower should pay back to the lender for the loan, $r$ is the simple interest rate, $t$ is the lending period (usually expressed in years), $P$ is the premium and $n$ is the number of times we compound the rate per year.
\end{definition}

Hence based on the value of $n$ we can have different compounded periods of intrayear compound interest rates:\\

\begin{itemize}
\item Annually: $n=1$
\item Semi-annually: $n=2$
\item Monthly: $n=12$
\item Weekly: $n=52$
\item Daily: $n=365$\\
\end{itemize}

One quantity that simplifies the complexity coming from intrayear compound interest rate is th so called \say{effective annual interest rate}:

\begin{definition}[Effective Annual Interest Rate]
\textbf{Effective annual interest rate} (EAR) $r_{\text{eff}}$ is the interest rate that is paid on an investment due to the result of intrayear compounding interest rate:

\begin{equation}
\Big( 1+\frac{r}{n} \Big)^{nt} \cdot P = (1+r_{\text{eff}}) \cdot P \:\:\: \Rightarrow \:\:\: r_{\text{eff}} = \Big( 1+\frac{r}{n} \Big)^{nt} - 1
\end{equation}
\end{definition}

\vspace{10pt}

In the extreme case where we compound the interest rate continuously (i.e for infinitesimally small amounts of time) we end up with infinite amount of compounds per year (i.e $n \to \infty$). This is the case of \say{continuously compound interest rate}:

\begin{definition}[Continuously Compound Interest Rate]
\textbf{Continuously compound interest rate} is a fixed rate $r$ applied on both the principle originally lent to the borrower and the compounding interest paid on the loan, where the compound takes place continuously throughout the year:

\begin{equation}
A = \lim_{n \to \infty} \Big( 1+\frac{r}{n} \Big)^{nt} \cdot P = \Big[ \lim_{n \to \infty} \Big( 1+\frac{r}{n} \Big)^n \Big]^t \cdot P = (e^r)^t \cdot P = e^{rt} \cdot P
\end{equation}

\vspace{10pt}

where in the third step we used the definition of $e$:

\begin{equation}
e^r =  \lim_{n \to \infty} \Big( 1+\frac{r}{n} \Big)^n
\end{equation}
\end{definition}

\section{Time Value Of Money}

Given that one can always invest money (even in a risk-free interest rate), it follows that receiving an $x$ amount of money today in not the same as receiving the same amount $x$ in the future, since by investing $x$ today in the future one would have both the amount $x$ and the interest on it. This idea introduces the concept of \say{present value} and \say{future value}.

\begin{definition}[Present Value]
\textbf{Present Value} $PV$ is the current worth of a future sum of money, or stream of cash flows, $FV$ after being discounted back to the present given a discount rate $r$:

\begin{equation}
PV = \frac{FV}{\Big(1+\frac{r}{n} \Big)^{nt}}
\end{equation}

\vspace{10pt} 

where $PV$ is the present value of a future payment, FV is the future value of a present payment, $r$ is the interest rate, $t$ is the lending period (usually expressed in years), and $n$ is the number of times we compound the rate per year.
\end{definition}

\begin{definition}[Future Value]
\textbf{Future Value} $FV$ is the future worth of a future sum of money, or stream of cash flows, $PV$ after being compounded to the future given a discount rate $r$:

\begin{equation}
FV = \Big(1+\frac{r}{n} \Big)^{nt} \cdot PV
\end{equation}

\vspace{10pt} 

where $PV$ is the present value of a future payment, FV is the future value of a present payment, $r$ is the interest rate, $t$ is the lending period (usually expressed in years), and $n$ is the number of times we compound the rate per year.
\end{definition}

It follows that a payment into the future might have different values since for different values of rate $r$ and compounding periods $n$ we end up with different kinds of present and future values.\\

More often than not in financial world instead of just one payment at a certain point in the future, we are dealing with a series of payments (called cash flows) in different future times. In general this is the concept of an \say{annuity}, however as we will see depending on the future periods we end up with different alternatives of an annuity.

\begin{definition}[Annuity]
\textbf{Annuity} is a series of equal payments (cash flows) for a number of future periods.
\end{definition}

In general the cash flows of an annuity can also be different to each other. However for now we will assume that we are dealing with the case where all future cash flows are equal to each other.

\begin{definition}[Ordinary Annuity]
\textbf{Ordinary annuity} is an annuity where the first payments takes place at $t=1$.
\end{definition}

\begin{figure}[H]
\includegraphics[scale=0.6]{annuity}
\centering
\caption {Ordinary Annuity}
\end{figure}

\vspace{10pt} 

We can calculate the present value of an ordinary annuity in the following way: since an ordinary annuity is a series of equal payments $CF_{1} = CF_{2} = \ldots = CF_{N}  = CF$ that take place in the future, its present value is the sum of all these payment, each discounted back to the present with a corresponding discounted factor based on the time period of the payment:

{\setlength{\jot}{15pt}
\begin{align*}
PV & = \frac{CF_1}{\Big(1+\frac{r}{n} \Big)^{n}} + \frac{CF_2}{\Big(1+\frac{r}{n} \Big)^{2n}} + \ldots +  \frac{CF_N}{\Big(1+\frac{r}{n} \Big)^{Nn}} \\
&=  \frac{CF}{\Big(1+\frac{r}{n} \Big)^{n}} + \frac{CF}{\Big(1+\frac{r}{n} \Big)^{2n}} + \ldots +  \frac{CF}{\Big(1+\frac{r}{n} \Big)^{Nn}} \\
&= CF \cdot \left[ \frac{1}{\Big(1+\frac{r}{n} \Big)^{n}} + \frac{1}{\Big(1+\frac{r}{n} \Big)^{2n}} + \ldots +  \frac{1}{\Big(1+\frac{r}{n} \Big)^{Nn}} \right] \\
&= CF \cdot \sum_{k=1}^{N} \left[ \left(1+\frac{r}{n} \right)^{-kn}  \right] \\
&= CF \cdot \left( \sum_{k=1}^{N} \left[ \left(1+\frac{r}{n} \right)^{k}  \right] \right)^{-n}
\end{align*}}

\vspace{10pt} 

By using the fact that the sum of the first n terms of a geometric series is:

\begin{equation*}
\sum _{k=0}^{n-1} x^{k} = \frac {1-x^{n}}{1-x}
\end{equation*}

\vspace{10pt} 

and by doing some very basic algebra we can show:

\begin{equation}
PV = CF \cdot \frac{1 - \left( 1 + \frac{r}{n} \right)^{-nN}}{\frac{r}{n}}
\end{equation}

\vspace{10pt} 

Following exactly the same steps but in the opposite order (i.e compound instead of discount), we can show that the future value of an ordinary annuity is:

\begin{equation}
PV = CF \cdot \frac{\left( 1 + \frac{r}{n} \right)^{nN} - 1}{\frac{r}{n}}
\end{equation}

\vspace{5pt} 

\begin{definition}[Annuity Due]
\textbf{Annuity due} is an annuity where the first payments takes place immediately at $t=0$.
\end{definition}

\begin{figure}[H]
\includegraphics[scale=0.6]{annuitydue}
\centering
\caption {Annuity Due}
\end{figure}

Calculating the present value of annuity due is very easy since all we have to do is to discount the present value of an ordinary annuity to the present $t=0$:

\begin{equation}
PV = CF \cdot \frac{1 - \left( 1 + \frac{r}{n} \right)^{-nN}}{\frac{r}{n}} \cdot \frac{1}{1 + \frac{r}{n}}
\end{equation}

\vspace{10pt} 

Similarly for the future value of an annuity due:

\begin{equation}
PV = CF \cdot \frac{\left( 1 + \frac{r}{n} \right)^{nN} - 1}{\frac{r}{n}} \cdot (1 + rn)
\end{equation}

\vspace{10pt} 

\begin{definition}[Perpetuity]
\textbf{Perpetuity} is an annuity with an infinite amount of future payments $N \to \infty$. 
\end{definition}

\begin{definition}[Ordinary Perpetuity]
\textbf{Ordinary perpetuity} is a perpetuity where the first payments takes place at $t=1$.
\end{definition}

The present value of an ordinary perpetuity can be calculated by setting $N \to \infty$ to the present value of an ordinary annuity:

\begin{equation}
PV = \lim_{N \to \infty} \left[ CF \cdot \frac{1 - \left( 1 + \frac{r}{n} \right)^{-nN}}{\frac{r}{n}} \right] = \frac{CF}{\frac{r}{n}}
\end{equation}

\vspace{10pt} 

The future value of an ordinary perpetuity does not make sense, since it blows up to infinity due to the infinite amount of payments.

\begin{definition}[Perpetuity due]
\textbf{Perpetuity due} is a perpetuity where the first payments takes place immediately at $t=0$.
\end{definition}

The present value of a perpetuity due can be calculated by setting $N \to \infty$ to the present value of an annuity due or even simpler by discount an ordinary annuity back to the present $t=0$:

\begin{equation}
PV = \frac{CF}{\frac{r}{n}} \cdot \frac{1}{\left(1+\frac{r}{n}\right)}
\end{equation}

\vspace{10pt} 

The future value of a perpetuity due does not make sense, since it blows up to infinity due to the infinite amount of payments.

\begin{definition}[Net Present Value]
\textbf{Net present value} (NPV) is the difference between the present value of an annuity and the amount of investment on the annuity:

\begin{equation}
NPV = \sum_{t=1}^{T} \frac{CF_t}{\Big(1+\frac{r}{n} \Big)^{nt}} - C_0
\end{equation}

\vspace{10pt} 

where we let the future payments of the annuity $CF_t$ to be different to each other, and $C_0$ is the initial investment on the annuity (undiscounted since it happened the day the annuity started).
\end{definition}

NPV is used in capital budgeting for analysing the profitability of a project. It follows that that a positive NPV means a project that will bring some profit while a negative NPV the opposite.

\begin{definition}[Internal Rate Of Return]
\textbf{Internal rate of return} (IRR) or, money weighted rate of return (MWRR), is the interest rate that makes the NPV of a project equal to zero:

\begin{equation}
\sum_{t=1}^{T} \frac{CF_t}{\Big(1+\frac{IRR}{n} \Big)^{nt}} - C_0 = 0
\end{equation}
\end{definition}

\vspace{10pt} 

The equation for IRR is not linear and there is no close analytical solution to obtain IRR. We can either usually try end error values or use numerical approaches to obtain it. IRR is used in order to measure the profitability of a future investment.